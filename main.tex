\documentclass[a4paper,12pt]{article}

% Pakete für zusätzliche Funktionen
\usepackage[utf8]{inputenc}   % UTF-8 Zeichencodierung
\usepackage[T1]{fontenc}     % Korrekte Silbentrennung und Umlaute
\usepackage[ngerman]{babel}  % Deutsche Spracheinstellungen
\usepackage{amsmath, amssymb} % Mathematische Symbole
\usepackage{graphicx}        % Einfügen von Bildern
\usepackage{hyperref}        % Hyperlinks
\usepackage{geometry}        % Seitenränder
\geometry{a4paper, margin=2.5cm}
\usepackage{listings}        % Code-Listings
\usepackage{csquotes}        % Zitate

% Einstellungen für Hyperlinks
\hypersetup{
    colorlinks=true,
    linkcolor=blue,
    citecolor=blue,
    filecolor=magenta,
    urlcolor=cyan,
    pdftitle={Implementierung digitaler Geschäftsprozesse},
    pdfpagemode=FullScreen,
}

% Titel und Autoreninformationen
\title{\textbf{Implementierung digitaler Geschäftsprozesse}}
\author{Kürsat Darcan | MFWS422A}
\date{Abgabedatum: \today}

\begin{document}

% Deckblatt
\maketitle
\thispagestyle{empty}
\vspace{2cm}
\begin{center}
    \includegraphics[width=0.3\textwidth]{FHDW_Logo_RGB-01.svg.png} % Ersetzen Sie "example-image" durch Ihr Logo/Bild
    \\
    \vspace{1cm}
    \textbf{Studiengang: Wirtschaftsinformatik}\\
    \textbf{Fachhochschule der Wirtschaft (FHDW)}
\end{center}
\newpage

% Inhaltsverzeichnis
\renewcommand{\thepage}{\roman{page}} % Seitenzahlen in römischen Ziffern für Verzeichnisse
\tableofcontents
\newpage

% Bei Bedarf Abbildungsverzeichnis
\listoffigures
\addcontentsline{toc}{section}{Abbildungsverzeichnis}
\newpage

% Bei Bedarf Tabellenverzeichnis
\listoftables
\addcontentsline{toc}{section}{Tabellenverzeichnis}
\newpage

% Abkürzungsverzeichnis
\section*{Abkürzungsverzeichnis}
\addcontentsline{toc}{section}{Abkürzungsverzeichnis}
\begin{description}
    \item[HTML] Hypertext Markup Language
    \item[CSS] Cascading Style Sheets
    \item[JS] JavaScript
\end{description}
\newpage

% Hauptteil mit arabischen Seitenzahlen
\renewcommand{\thepage}{\arabic{page}} % Seitenzahlen in arabischen Ziffern für den Hauptteil
\setcounter{page}{1}


% Einleitung
\section{Einleitung}
\subsection{Zielsetzung der Ausarbeitung}
\subsection{Überblick über das Planspiel kdibisglobal}
\subsection{Rolle des Unternehmens: Einzelhandel im Biermarkt}

% Ablauf und Reflexion des Planspiels
\section{Ablauf und Reflexion des Planspiels}

\subsection{Spielrunde 1 – SCM1: Bullwhip Game und ERP-Strategie}
\subsubsection{Analyse des eigenen Bestellverhaltens im Einzelhandel}
\subsubsection{Ursachen des Bullwhip-Effekts im Kontext des Einzelhandels}
\subsubsection{Distributionslogistik und Bestellmengen im Einzelhandel}
\subsubsection{IT-Situation der Einzelhandelsketten 1–4}
\subsubsection{Wahl einer M\&A IT-Integrationsstrategie}
\subsubsection{Organisational Readiness und CMMI-Reifegrade}

\subsection{Spielrunde 2 – SCM2: Forecasting und Inventory Management}
\subsubsection{Bestell- und Lieferverhalten im Einzelhandel}
\subsubsection{Teamstrategie und interne Abstimmung}
\subsubsection{Auswahl und Anwendung von Forecastingmethoden}
\subsubsection{Optimierung der Bestellkosten im Einzelhandel}
\subsubsection{Umgang mit Lieferverzögerungen über Blockchain \& Smart Contracts}

\subsection{Spielrunde 3 – CRM3: Kundenmanagement mit Big Data}
\subsubsection{Analyse der Einzelhandels-Ergebnisse im 3. Fiskaljahr}
\subsubsection{ Performance-Analyse mit Word Tree \& beworbenen Produkten}
\subsubsection{Einsatz von Sentiment Analysis im CRM und Marketing}
% Fazit
\section{Fazit}
\subsection{Wichtige Erkenntnisse für den Einzelhandel aus den Spielrunden}
\subsection{Bewertung des eigenen Verhaltens und der Teamkoordination}
% Literaturverzeichnis
\newpage
\addcontentsline{toc}{section}{Literaturverzeichnis}
\section*{Literaturverzeichnis}
\begin{thebibliography}{99}
    \bibitem{Schlottke2020} Schlottke, F., Krcmar, H. (2020). Einführung in die Digitalisierung. In: Fähnrich, K.P., Franczyk, B. (Hrsg.) \textit{Digitale Geschäftsprozesse}, pp. 1–24. Springer Vieweg, Berlin, Heidelberg. Verfügbar unter: \url{https://link.springer.com/chapter/10.1007/978-3-662-59815-3_1} (zuletzt aufgerufen am 23.04.2025).
\end{thebibliography}


\newpage
% Anhang
\appendix
\section{Anhang}
Hier können zusätzliche Informationen, wie Code-Beispiele oder ausführliche Tabellen, eingefügt werden.

% Ehrenwörtliche Erklärung
\newpage
\addcontentsline{toc}{section}{Ehrenwörtliche Erklärung}
\section*{\texttt{Ehrenwörtliche Erklärung}}
Hiermit erkläre ich, dass ich die vorliegende schriftliche Ausarbeitung im Modul \textbf{Implementierung digitaler Geschäftsprozesse} selbstständig
angefertigt habe. Es wurden nur die in der Arbeit ausdrücklich benannten Quellen und
Hilfsmittel benutzt. Wörtlich oder sinngemäß übernommenes Gedankengut habe ich als
solches kenntlich gemacht. Diese Arbeit hat in gleicher oder ähnlicher Form noch keiner
Prüfungsbehörde vorgelegen.

\vspace{3cm}
\noindent\begin{tabular}{p{0.5\textwidth}p{0.5\textwidth}}
    \hrulefill & \hrulefill \\
    Ort, Datum & Unterschrift \\
\end{tabular}

\end{document}
