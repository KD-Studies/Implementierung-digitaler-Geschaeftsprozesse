\documentclass[a4paper,12pt]{article}

% Pakete für zusätzliche Funktionen
\usepackage[utf8]{inputenc}   % UTF-8 Zeichencodierung
\usepackage[T1]{fontenc}     % Korrekte Silbentrennung und Umlaute
\usepackage[ngerman]{babel}  % Deutsche Spracheinstellungen
\usepackage{amsmath, amssymb} % Mathematische Symbole
\usepackage{graphicx}        % Einfügen von Bildern
\usepackage{hyperref}        % Hyperlinks
\usepackage{geometry}        % Seitenränder
\geometry{a4paper, margin=2.5cm}
\usepackage{listings}        % Code-Listings
\usepackage{csquotes}        % Zitate

% Einstellungen für Hyperlinks
\hypersetup{
    colorlinks=true,
    linkcolor=blue,
    citecolor=blue,
    filecolor=magenta,
    urlcolor=cyan,
    pdftitle={Implementierung digitaler Geschäftsprozesse},
    pdfpagemode=FullScreen,
}

% Titel und Autoreninformationen
\title{\textbf{Implementierung digitaler Geschäftsprozesse}}
\author{Kürsat Darcan | MFWS422A}
\date{Abgabedatum: \today}

\begin{document}

% Deckblatt
\maketitle
\thispagestyle{empty}
\vspace{2cm}
\begin{center}
    \includegraphics[width=0.3\textwidth]{FHDW_Logo_RGB-01.svg.png} % Ersetzen Sie "example-image" durch Ihr Logo/Bild
    \\
    \vspace{1cm}
    \textbf{Studiengang: Wirtschaftsinformatik}\\
    \textbf{Fachhochschule der Wirtschaft (FHDW)}
\end{center}
\newpage

% Inhaltsverzeichnis
\renewcommand{\thepage}{\roman{page}} % Seitenzahlen in römischen Ziffern für Verzeichnisse
\tableofcontents
\newpage

% Bei Bedarf Abbildungsverzeichnis
\listoffigures
\addcontentsline{toc}{section}{Abbildungsverzeichnis}
\newpage

% Bei Bedarf Tabellenverzeichnis
\listoftables
\addcontentsline{toc}{section}{Tabellenverzeichnis}
\newpage

% Abkürzungsverzeichnis
\section*{Abkürzungsverzeichnis}
\addcontentsline{toc}{section}{Abkürzungsverzeichnis}
\begin{description}
    \item[HTML] Hypertext Markup Language
    \item[CSS] Cascading Style Sheets
    \item[JS] JavaScript
\end{description}
\newpage

% Hauptteil mit arabischen Seitenzahlen
\renewcommand{\thepage}{\arabic{page}} % Seitenzahlen in arabischen Ziffern für den Hauptteil
\setcounter{page}{1}


% Einleitung
\section{Einleitung}
Die Einleitung stellt das Thema der Arbeit vor und gibt eine Übersicht über den Aufbau der Arbeit. Sie erklärt, warum das Thema relevant ist und welche Fragestellungen bearbeitet werden.

% Hauptteil
\section{Hauptteil}
\subsection{Grundlagen}
Hier werden die theoretischen Grundlagen beschrieben, die für das Verständnis der Arbeit notwendig sind.

\subsection{Technische Aspekte}
Dieser Abschnitt beschäftigt sich mit den technischen Details der untersuchten Technologien.

\subsection{Anwendungen}
Hier werden praktische Anwendungen beschrieben und analysiert.

% Fazit
\section{Fazit und Ausblick}
Das Fazit fasst die Ergebnisse der Arbeit zusammen und gibt einen Ausblick auf zukünftige Entwicklungen oder Forschungsthemen.

% Literaturverzeichnis
\newpage
\addcontentsline{toc}{section}{Literaturverzeichnis}
\section*{Literaturverzeichnis}
\begin{thebibliography}{99}
    \bibitem{Beispiel1} Beispielautor, B. (2025). \textit{Titel des Buches}. Verlag, Ort.
    \bibitem{Beispiel2} Beispielautor, C. und Beispielautor, D. (2024). Artikelname. \textit{Zeitschrift}, 12(3), 45-67.
    \bibitem{Beispiel3} Beispielautor, E. (2023). Online-Ressource. Verfügbar unter: \url{https://www.example.com} (zuletzt aufgerufen am 01.01.2025).
\end{thebibliography}

\newpage
% Anhang
\appendix
\section{Anhang}
Hier können zusätzliche Informationen, wie Code-Beispiele oder ausführliche Tabellen, eingefügt werden.

% Ehrenwörtliche Erklärung
\newpage
\addcontentsline{toc}{section}{Ehrenwörtliche Erklärung}
\section*{\texttt{Ehrenwörtliche Erklärung}}
Hiermit erkläre ich, dass ich die vorliegende schriftliche Ausarbeitung im Modul \textbf{Webtechnologies and Applications} selbstständig
angefertigt habe. Es wurden nur die in der Arbeit ausdrücklich benannten Quellen und
Hilfsmittel benutzt. Wörtlich oder sinngemäß übernommenes Gedankengut habe ich als
solches kenntlich gemacht. Diese Arbeit hat in gleicher oder ähnlicher Form noch keiner
Prüfungsbehörde vorgelegen.

\vspace{3cm}
\noindent\begin{tabular}{p{0.5\textwidth}p{0.5\textwidth}}
    \hrulefill & \hrulefill \\
    Ort, Datum & Unterschrift \\
\end{tabular}

\end{document}
