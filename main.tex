\documentclass[a4paper,12pt]{article}

% Pakete für zusätzliche Funktionen
\usepackage[utf8]{inputenc}   % UTF-8 Zeichencodierung
\usepackage[T1]{fontenc}     % Korrekte Silbentrennung und Umlaute
\usepackage[ngerman]{babel}  % Deutsche Spracheinstellungen
\usepackage{amsmath, amssymb} % Mathematische Symbole
\usepackage{graphicx}        % Einfügen von Bildern
\usepackage{hyperref}        % Hyperlinks
\usepackage{geometry}        % Seitenränder
\usepackage{listings}        % Code-Listings
\usepackage{csquotes}        % Zitate
\usepackage{csvsimple}  % Für das Einfügen von CSV-Dateien

\geometry{a4paper, margin=2.5cm}
% Einstellungen für Hyperlinks
\hypersetup{
    colorlinks=true,
    linkcolor=blue,
    citecolor=blue,
    filecolor=magenta,
    urlcolor=cyan,
    pdftitle={Implementierung digitaler Geschäftsprozesse},
    pdfpagemode=FullScreen,
}

% Titel und Autoreninformationen
\title{\textbf{Implementierung digitaler Geschäftsprozesse}}
\author{Kürsat Darcan | MFWS422A}
\date{Abgabedatum: \today}

\begin{document}

% Deckblatt
\maketitle
\thispagestyle{empty}
\vspace{2cm}
\begin{center}
    \includegraphics[width=0.3\textwidth]{FHDW_Logo_RGB-01.svg.png} % Ersetzen Sie "example-image" durch Ihr Logo/Bild
    \\
    \vspace{1cm}
    \textbf{Studiengang: Wirtschaftsinformatik}\\
    \textbf{Fachhochschule der Wirtschaft (FHDW)}
\end{center}
\newpage

% Inhaltsverzeichnis
\renewcommand{\thepage}{\roman{page}} % Seitenzahlen in römischen Ziffern für Verzeichnisse
\tableofcontents
\newpage

% Bei Bedarf Abbildungsverzeichnis
\listoffigures
\addcontentsline{toc}{section}{Abbildungsverzeichnis}
\newpage

% Bei Bedarf Tabellenverzeichnis
\listoftables
\addcontentsline{toc}{section}{Tabellenverzeichnis}
\newpage
%-- Beispiel für eine Tabelle --%
%\begin{table}[htbp]
%    \centering
%    \resizebox{\textwidth}{!}{ 
%    \csvautotabular[separator=semicolon]{tabellen/scm1/test.csv}  % Pfad zur CSV-Datei
%    }
%    \caption{Beispieltabelle aus test.csv}
%    \label{tab:testcsv}
%\end{table}
%-- Ende Beispiel --%

% Abkürzungsverzeichnis
\section*{Abkürzungsverzeichnis}
\addcontentsline{toc}{section}{Abkürzungsverzeichnis}
\begin{description}
    \item[CRM] Customer Relationship Management
    \item[SCM] Supply Chain Management
\end{description}
\newpage

% Hauptteil mit arabischen Seitenzahlen
\renewcommand{\thepage}{\arabic{page}} % Seitenzahlen in arabischen Ziffern für den Hauptteil
\setcounter{page}{1}


% Einleitung
\section{Einleitung}
\subsection{Zielsetzung der Ausarbeitung}
Diese Ausarbeitung ist Teil des Moduls \textbf{Implementierung digitaler Geschäftsprozesse}.
Die Ausarbeitung bildet den Ablauf und die Reflexion des Planspiels \textit{kdibisglobal} ab,
das im Rahmen des Moduls durchgeführt wurde. Hierbei wird auf die einzelnen Spielrunden eingegangen
und die jeweiligen Ergebnisse und Erkenntnisse analysiert und reflektiert. Zusätzlich werden weitere Methoden
erläutert, die im Rahmen des Moduls behandelt wurden, aber nicht im Planspiel angewendet werden konnten.

\subsection{Überblick über das Planspiel kdibisglobal}
\textbf{kdibisglobal} wurde speziell für das Buch Integrierte Business-Informationssysteme von Herrn Klaus-Dieter Gronwald entwickelt,
um ein praktisches Verständnis für digitale Geschäftsprozesse zu erlangen.
Das Planspiel simuliert das Geschäftsprozessmanagement für einen Bierhersteller. 
In diesem Planspiel übernehmen die Teilnehmer einzelne Bereiche innerhalb des Unternehmens und sind für die jeweiligen Bestellungen zuständig. 
Dabei müssen verschiedene Faktoren berücksichtigt werden, wie Lieferverzögerungen, Verfügbarkeit von Produkten, saisonale Nachfrage und so weiter.
Zusätzlich werden wichtige Bereiche wie Supply Chain Management (SCM) und Customer Relationship Management (CRM) abgebildet.
Im Rahmen des SCM geht es darum, Bestände sinnvoll zu planen, Nachbestellungen rechtzeitig auszulösen und Lieferengpässe zu vermeiden.
Beim CRM steht die Beziehung zum Kunden im Vordergrund, also etwa das Management von Aufträgen,
die Sicherstellung einer hohen Kundenzufriedenheit sowie die Reaktion auf Änderungen in der Nachfrage.
Ziel ist es, durch den richtigen Einsatz digitaler Systeme die Unternehmensprozesse effizient zu gestalten \cite{Kdibisglobal2025}

% Ablauf und Reflexion des Planspiels
\section{Ablauf und Reflexion des Planspiels}

\subsection{Spielrunde 1 – SCM1: Bullwhip Game und ERP-Strategie}


\subsubsection{Analyse des eigenen Bestellverhaltens im Einzelhandel}
\subsubsection{Ursachen des Bullwhip-Effekts im Kontext des Einzelhandels}
\subsubsection{Distributionslogistik und Bestellmengen im Einzelhandel}
\subsubsection{IT-Situation der Einzelhandelsketten 1–4}
\subsubsection{Wahl einer M\&A IT-Integrationsstrategie}
\subsubsection{Organisational Readiness und CMMI-Reifegrade}

\subsection{Spielrunde 2 – SCM2: Forecasting und Inventory Management}
\subsubsection{Bestell- und Lieferverhalten im Einzelhandel}
\subsubsection{Teamstrategie und interne Abstimmung}
\subsubsection{Auswahl und Anwendung von Forecastingmethoden}
\subsubsection{Optimierung der Bestellkosten im Einzelhandel}
\subsubsection{Umgang mit Lieferverzögerungen über Blockchain \& Smart Contracts}
\subsubsection{Anwendung des Kanban-Prinzips zur Optimierung der Lieferkette}

\subsection{Spielrunde 3 – CRM3: Kundenmanagement mit Big Data}
\subsubsection{Analyse der Einzelhandels-Ergebnisse im 3. Fiskaljahr}
\subsubsection{ Performance-Analyse mit Word Tree \& beworbenen Produkten}
\subsubsection{Einsatz von Sentiment Analysis im CRM und Marketing}

% Fazit
\section{Fazit}
\subsection{Wichtige Erkenntnisse für den Einzelhandel aus den Spielrunden}
\subsection{Bewertung des eigenen Verhaltens und der Teamkoordination}

% Literaturverzeichnis
\newpage
\addcontentsline{toc}{section}{Literaturverzeichnis}
\section*{Literaturverzeichnis}
\begin{thebibliography}{99}
    \bibitem{Schlottke2020} Schlottke, F., Krcmar, H. (2020). Einführung in die Digitalisierung. In: Fähnrich, K.P., Franczyk, B. (Hrsg.) \textit{Digitale Geschäftsprozesse}, Springer Vieweg, Berlin, Heidelberg. Verfügbar unter: \url{https://link.springer.com/book/10.1007/978-3-662-59815-3} (zuletzt aufgerufen am 27.04.2025).
    \bibitem{Kdibisglobal2025} Gronwald, K.D. \textit{kdibisglobal – Planspiel zur Umsetzung integrierter Business-Informationssysteme}. Verfügbar unter: \url{https://www.kdibisglobal.org/php/kdiglobstart.php} (zuletzt aufgerufen am 27.04.2025)
\end{thebibliography}


\newpage
% Anhang
\appendix
\section{Anhang}
Hier können zusätzliche Informationen, wie Code-Beispiele oder ausführliche Tabellen, eingefügt werden.

% Ehrenwörtliche Erklärung
\newpage
\addcontentsline{toc}{section}{Ehrenwörtliche Erklärung}
\section*{\texttt{Ehrenwörtliche Erklärung}}
Hiermit erkläre ich, dass ich die vorliegende schriftliche Ausarbeitung im Modul \textbf{Implementierung digitaler Geschäftsprozesse} selbstständig
angefertigt habe. Es wurden nur die in der Arbeit ausdrücklich benannten Quellen und
Hilfsmittel benutzt. Wörtlich oder sinngemäß übernommenes Gedankengut habe ich als
solches kenntlich gemacht. Diese Arbeit hat in gleicher oder ähnlicher Form noch keiner
Prüfungsbehörde vorgelegen.

\vspace{3cm}
\noindent\begin{tabular}{p{0.5\textwidth}p{0.5\textwidth}}
    \hrulefill & \hrulefill \\
    Ort, Datum & Unterschrift \\
\end{tabular}

\end{document}
